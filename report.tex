\documentclass[a4paper]{article}
\usepackage[UTF8]{ctex}
\usepackage[left=3.17cm, right=3.17cm, top=2.54cm, bottom=2.54cm]{geometry}
\usepackage[dvipsnames]{xcolor}
\usepackage[most]{tcolorbox}
\usepackage{fancyhdr}
\usepackage{graphicx, subfigure}
\usepackage{minted}
\usepackage{amsmath}
\usepackage[shortlabels]{enumitem}
\setmonofont[]{Fira Code}
\setenumerate{itemsep=0pt,partopsep=0pt,parsep=\parskip,topsep=0pt, itemindent=4em, leftmargin=0pt, listparindent=2em, label=(\arabic*)}
\setitemize{itemsep=0pt,partopsep=0pt,parsep=\parskip,topsep=5pt}
\setdescription{itemsep=0pt,partopsep=0pt,parsep=\parskip,topsep=5pt}

\usepackage[sort&compress]{gbt7714}
\usepackage{hyperref}
\hypersetup{unicode, colorlinks=true, linkcolor=black, urlcolor=black}
\usepackage{booktabs}
\usepackage{multirow}
\usepackage{color}
\usepackage{listings}
\usepackage{fontspec}

\definecolor{codegreen}{rgb}{0,0.6,0}
\definecolor{codegray}{rgb}{0.5,0.5,0.5}
\definecolor{codepurple}{rgb}{0.58,0,0.82}
\definecolor{backcolour}{rgb}{0.95,0.95,0.95}

\lstdefinestyle{mystyle}{
    backgroundcolor=\color{backcolour},   
    commentstyle=\color{codegreen},
    keywordstyle=\color{blue},
    numberstyle=\tiny\color{codegray},
    stringstyle=\color{codepurple},
    basicstyle=\ttfamily\footnotesize,
    breakatwhitespace=false,         
    breaklines=true,                 
    captionpos=b,                    
    keepspaces=true,                 
    numbers=left,                    
    numbersep=5pt,                  
    showspaces=false,                
    showstringspaces=false,
    showtabs=false,                  
    tabsize=2
}

\lstset{style=mystyle}

\begin{document}

% \definecolor{bg}{rgb}{0.98,0.98,0.98}

\title{\textbf{MATLAB大作业之图像处理}}

\author{无03\quad 王与进 \quad2020010708}
\date{}

\maketitle

\tableofcontents

\newpage

\section{文件列表}
\begin{lstlisting}[language=matlab, caption=文件列表]
.
└── code
    ├── data % 测试用数据
    │   ├── Faces/
    │   ├── hall.mat
    │   ├── JpegCoeff.mat
    │   ├── snow.mat
    │   ├── test_1.bmp
    │   ├── test_2.bmp
    │   ├── test_3.bmp
    │   ├── test_4.bmp
    │   └── test_5.bmp
    ├── ac_decode.m % ac解码 % --- 以下为函数 ---
    ├── ac_encode.m % ac编码
    ├── ac_huffman_tree_dict.m % 获取ac编码的huffman tree字典
    ├── binary_decode.m % 解码二进制编码
    ├── binary_encode.m % 编码二进制编码
    ├── calculate_compress_ratio.m % 计算压缩比
    ├── calculate_correct_ratio.m % 计算解码信息正确率
    ├── concact_rgb.m % 拼接rgb值
    ├── dc_decode.m % dc解码
    ├── dc_encode.m % dc编码
    ├── dc_huffman_tree_dict.m % 获取dc编码的huffman tree字典
    ├── extract_face_feature.m % 提取训练集特征
    ├── get_dct_matrix.m % 获取dct值
    ├── get_dct_matrix_with_params.m
    ├── get_dct_params.m
    ├── get_face_area.m % 识别脸部方框
    ├── get_probability_density.m % 获取图片颜色概率密度
    ├── huffman_encode.m % huffman编码
    ├── inv_zig_zag.m % 反zigzag
    ├── jpeg_decode.m % jpeg解码
    ├── jpeg_decode_with_info.m % jpeg解码(带有隐藏信息)
    ├── jpeg_decode_with_params.m
    ├── jpeg_encode.m % jpeg编码
    ├── jpeg_encode_with_info.m % jpeg编码(带有隐藏信息)
    ├── jpeg_encode_with_params.m
    ├── my_dct.m % dct算法实现
    ├── psnr.m % 计算PSNR
    ├── run_size_encode.m % run/size编码
    ├── test_ac_encode.m % ac编码测试
    ├── zig_zag.m % zig-zag扫描
    ├── hw_1_1.m % --- 以下为作业脚本 ---
    ├── hw_1_2.m
    ├── hw_2_1.m
    ├── hw_2_2.m
    ├── hw_2_3.m
    ├── hw_2_4.m
    ├── hw_2_5.m
    ├── hw_2_7.m
    ├── hw_2_8.m
    ├── hw_2_9.m
    ├── hw_2_11.m
    ├── hw_2_12.m
    ├── hw_2_13.m
    ├── hw_3_1.m
    ├── hw_3_2.m
    ├── hw_4_1.m
    ├── hw_4_2_extra.m
    ├── hw_4_2.m % 4-2分为2个文件
    ├── hw_4_3_1.m
    ├── hw_4_3_2.m
    └── hw_4_3_3.m % 4-3分为3个文件
    
\end{lstlisting}

\section{练习题}

\subsection{基础知识}
\subsubsection{图像处理工具箱}
\par 在MATLAB终端中输入help images:

\begin{figure}[ht]
     \centering
     \includegraphics[width=.8\textwidth]{images.jpg}
     \caption{help images}
    \label{fig:help images}
\end{figure}

\subsubsection{简单的图像处理}
\begin{description}
    \item[绘制红色圆形] 
    \par 在本例中,我使用了matlab自带的rectangle函数。其中width/2-r和height/2-r表示圆外切正方形的左下端的坐标,2*r表示圆的直径。
    \begin{lstlisting}[language=matlab, caption=绘制红色圆形]
    load('data/hall.mat')
    hall_with_circle = hall_color;
    imshow(hall_with_circle);

    % get size
    [height, width, ~] = size(hall_color);
    r = min(height, width)/2;
    hold on;

    % draw circle
    rectangle('position',[width/2-r,height/2-r,2*r,2*r],'curvature',[1,1],'EdgeColor','r');
    \end{lstlisting}
    
    \item[绘制黑白格]
    \par 在本例中,我们需要计算出每个像素点所在的区间。如果区间序号模2得到的结果不同,则将其涂黑。
    \begin{lstlisting}[language=matlab, caption=绘制棋盘]
    block_width = 8;
    height_blocks = height/block_width;
    width_blocks = width/block_width;
    for i = 1:height
        for j = 1:width
            if (xor(mod(floor(i/block_width), 2), mod(floor(j/block_width),2) == 1))
            for k = 1:3
                hall_with_chessboard(i,j,k) = 0;
                end
            end
        end
    end
    \end{lstlisting}
\end{description}

\par 最终结果如下:
\begin{figure}[ht]
     \centering
     \includegraphics[width=.8\textwidth]{result1.jpg}
     \caption{result 1}
    \label{fig:result 1}
\end{figure}

\subsection{图像压缩编码}

\subsubsection{图像的预处理}
\par 图像的预处理是将每个像素的灰度值减去128。这一变换可以在变换域中进行,理由如下:
\par 考虑二维DCT的矩阵形式:
\begin{equation}
    \mathbf{C} = \mathbf{D}\mathbf{P}\mathbf{D^T}
\end{equation}
\par 显然我们有:
\begin{equation}
    \mathbf{D}(\mathbf{P_1}+\mathbf{P_2})\mathbf{D^T} = \mathbf{D}\mathbf{P_1}\mathbf{D^T} + \mathbf{D}\mathbf{P_2}\mathbf{D^T}
\end{equation}
\par 即二维DCT是一个线性变换。我们不妨设$\mathbf{P_2} = 128\mathbf{J}$,其中$\mathbf{J}$为全1矩阵,大小为$N \times N$,则$\mathbf{J} = \overrightarrow{l}\overrightarrow{l}^T$,其中$ \overrightarrow{l}$为长度为1的全1向量。
\par 由DCT的定义可知:
\begin{equation}
    \mathbf{D} = \sqrt{\frac{2}{N}}\begin{pmatrix}
    \sqrt{\frac{1}{2}} & \cdots & \sqrt{\frac{1}{2}} \\
    \vdots & \ddots & \vdots \\
    \cos\frac{(N-1)\pi}{2N} & \cdots & \cos\frac{(N-1)(2N-1)\pi}{2N}
    \end{pmatrix}
\end{equation}
\par 考虑到$\Sigma_{i=0}^{N}\cos \frac{k(2i-1)}{2N} = 0$,显然$\mathbf{D}\overrightarrow{l} = [\sqrt{N}, 0, \cdots, 0]^T$,则我们可以得出:
\begin{equation}
    \mathbf{D}\mathbf{P_2}\mathbf{D^T} = \\
    128\mathbf{D}\overrightarrow{l}\overrightarrow{l}^T\mathbf{D^T} = \\
    128N\begin{pmatrix}
    1 & \cdots & 0 \\
    \vdots & \ddots & \vdots \\
    0 & \cdots & 0
    \end{pmatrix}
\end{equation}
\par 即$\mathbf{P_2}$的DCT如上式所示,记为$\mathbf{K}$。
\par 根据题意,我们只需要先将图像进行DCT变换,再将左上角位置的元素减去$128N$即可。为验证这一结论,我们从测试图片中截取8x8的一块进行计算,其中用到了MATLAB中内置的dct2函数:
\begin{lstlisting}[language=matlab, caption=两种DCT变换形式比较]
load('data/hall.mat')
test_img = double(hall_gray(1:8,1:8));

result_1 = dct2(test_img - 128);
result_2 = dct2(test_img);
result_2(1,1) = result_2(1,1) - 128*8;

disp(sum(sum(abs(result_1 - result_2))));
\end{lstlisting}
\par 可以得到两者差的绝对值之和为$9.0705\times10^{-13}$,这说明在误差范围之内,两种计算方法是等效的。

\subsubsection{编程实现二维DCT}
\par 二维DCT的代码见\lstinline{my_dct.m}。我们将DCT变换扩充到了非方阵情形:
\begin{lstlisting}[language=matlab, caption=自定义DCT]
function [C] = my_dct(P)
    [h, w] = size(P);
    D_1 = zeros(h,h);
    D_1(1,:) = 0.5^0.5;
    disp(h);
    for i = 2:h
        D_1(i,:) = cos(pi*(i-1)*(1:2:(2*h-1))/(2*h));
    end
    D_1 = D_1 * (2/h) ^ 0.5;
    
    D_2 = zeros(w,w);
    D_2(1,:) = 0.5^0.5;
    for i = 2:w
        D_2(i,:) = cos(pi*(i-1)*(1:2:(2*w-1))/(2*w));
    end
    D_2 = D_2 * (2/w) ^ 0.5;
    
    C = D_1 * double(P) * D_2';
end
\end{lstlisting}
\par 测试如下:
\begin{lstlisting}[language=matlab, caption=两种DCT变换形式比较]
load('data/hall.mat')
test_img = double(hall_gray(1:7,1:8));

result_1 = dct2(test_img);
result_2 = my_dct(test_img);

disp(result_1);
disp(result_2);

disp(sum(sum(abs(result_1 - result_2))));
\end{lstlisting}
\par 可以得到两者差的绝对值之和为$4.7719\times10^{-12}$,这说明在误差范围之内,两种计算方法是等效的。

\subsubsection{DCT系数置零}
\par 为方便起见,我们将生成DCT系数的过程单独抽取为一个函数\lstinline{dct_param.m}。
\par 在正式进行实验之前,我们需要探讨DCT系数的意义:在DCT系数中,右侧表示高频分量,左侧表示低频分量。更加具体地讲,左上侧系数表示图像的整体亮度,左下侧系数表示纵向变化的纹理的强度,右上侧系数表示横向变化的纹理的强度,右下侧系数表示横竖两个方向均迅速变化的成分。由于常见的图片以缓慢变化为主,所以DCT系数左方取值较大,右方取值较小。我们猜测:去除右侧4列,对图像呈现效果影响不大,但去除左侧4列影响会较大。
\par 验证代码如下:
\begin{lstlisting}[language=matlab, caption=DCT置零方式比较]
load('data/hall.mat')
P = double(hall_gray);
subplot(3,1,1);
imshow(uint8(P));
title('original');

% DCT with right
[height, width] = size(P);
D_1 = get_dct_params(height);
D_2 = get_dct_params(width);

C = D_1 * P *D_2';
C(:,end-3:end) = 0;

% IDCT with right

P1 = D_1' * C * D_2;
subplot(3,1,2);
imshow(uint8(P1));
title('right 0');

% DCT with left
C = D_1 *P *D_2';
C(:,1:4) = 0;

% IDCT with left

P2 = D_1' * C * D_2;
subplot(3,1,3);
imshow(uint8(P2));
title('left 0');
\end{lstlisting}

\par 结果如下:

\begin{figure}[ht]
     \centering
     \includegraphics[width=.6\textwidth]{result2.jpg}
     \caption{result 2}
    \label{fig:result 2}
\end{figure}

\par 可以看出,与原始图像相比,若将左侧系数置零,则图像受损严重且较暗;而将右侧系数置零,则图像几乎可以完全恢复,与预期结果一致。

\par 为了进一步分析两种变换对图像的影响,我们截取一部分进行分析:

\begin{figure}[ht]
     \centering
     \includegraphics[width=.6\textwidth]{result3.jpg}
     \caption{result 3}
    \label{fig:result 3}
\end{figure}

\par 可以得出以下结论:
\begin{enumerate}
    \item 若删除左侧系数,我们只能看到纵向的高频分量,而低频分量接近0。
    \item 若删除右侧系数,则纵向的高频分量被清除,纵向颜色变化较为平缓,但亮度几乎不变。
\end{enumerate}

\subsubsection{DCT系数的转置和旋转}
\par 代码如下:
\begin{lstlisting}[language=matlab, caption=DCT系数的转置和旋转]
load('data/hall.mat')
P = double(hall_gray(1:120,1:120));
subplot(1,4,1);
imshow(uint8(P));
title('original');

[height, width] = size(P);
D_1 = get_dct_params(height);
D_2 = get_dct_params(width);

% DCT with transpose
C = D_1 * P *D_2';

% IDCT with transpose
P1 = D_1' * C' * D_2;
subplot(1,4,2);
imshow(uint8(P1));
title('transpose');

% IDCT with rot 90
P2 = D_1' * rot90(C) * D_2;
subplot(1,4,3);
imshow(uint8(P2));
title('rot 90');

% IDCT with rot 180
P2 = D_1' * rot90(C,2) * D_2;
subplot(1,4,4);
imshow(uint8(P2));
title('rot 180');

\end{lstlisting}

\par 结果如下:
\begin{figure}[ht]
     \centering
     \includegraphics[width=.8\textwidth]{result4.jpg}
     \caption{result 4}
    \label{fig:result 4}
\end{figure}

\par 据此,我们可以得出以下结论:
\begin{enumerate}
    \item DCT系数转置后的逆变换即为转置后的图像,证明如下:
    \begin{equation}
        \mathbf{D}^T\mathbf{C}^T\mathbf{D} = (\mathbf{D}^T\mathbf{C}\mathbf{D})^T = \mathbf{P}^T
    \end{equation}
    \item DCT系数旋转90度后,原先右上角的系数将会移动到左上角,而原先左下角的系数将会移到右下角。这意味着,图片的横向纹理较强。
    \item DCT系数旋转180度后,原先右下角的系数将会移动到左上角。这意味着,图片的纹理会呈现出斜向变化。
\end{enumerate}

\subsubsection{差分编码的频率响应}
\par 由于差分编码的表达式可以写作:
\begin{equation}
    y(n) = x(n-1) - x(n)
\end{equation}
\par 可以用以下代码绘制其频率响应:

\begin{lstlisting}[language=matlab, caption=差分编码频响]
b = [-1 1];
a = 1;
freqz(b,a);
\end{lstlisting}

\begin{figure}[ht]
     \centering
     \includegraphics[width=.8\textwidth]{result5.jpg}
     \caption{频率响应}
    \label{fig:result 5}
\end{figure}

\par 由图可知,这是一个高通滤波器。DC系数的高频分量更多。

\subsubsection{DC error和category}
\begin{equation}
\begin{cases}
    Category = \lfloor \log_2( |error| )\rfloor + 1 , (error \neq 0) \\
    Category = 0 , (error = 0)
\end{cases}
\end{equation}

\subsubsection{zigzag算法的实现}
\par 观察发现zigzag算法的特点:
\begin{enumerate}
    \item 横纵坐标之和单调递增
    \item 扫描方向与横纵坐标的奇偶性有关
\end{enumerate}
\par 据此可以设计zigzag算法。zigzag算法见\lstinline{zig_zag.m}:
\begin{lstlisting}[language=matlab, caption=zigzag算法]
function [idx] = zig_zag(mat_size)
    idx = [];
    y = reshape(1:mat_size*mat_size, mat_size, mat_size);
    
    for i = 2:2*mat_size
        for j = max(i - mat_size, 1):min(mat_size, i-1)
            if mod(i, 2) == 0
               idx = [idx, y(i-j, j)];
            else
               idx = [idx, y(j, i-j)];
            end
        end
    end
end
\end{lstlisting}

\par 测试结果如下,可见算法正确:
\begin{figure}[ht]
     \centering
     \includegraphics[width=.8\textwidth]{result6.jpg}
     \caption{zigzag测试结果}
    \label{fig:result 6}
\end{figure}

\subsubsection{测试图像DCT、分块和量化}
\par DCT、分块和量化的步骤如下:
\begin{enumerate}
    \item 将图片分为8x8的若干小块。若图片的长和宽不能被8整除,则将其补全。
    \item 将每个小块预处理(-128)之后进行DCT变换。
    \item 将DCT变换后的矩阵进行zigzag扫描,并进行拼接。
\end{enumerate}
\begin{lstlisting}[language=matlab, caption=DCT、分块和量化]
load('data/hall.mat');
load('data/JpegCoeff');
P = double(hall_gray);

% fill the picture to block it
[h, w] = size(P);
if(mod(h, 8) ~=0)
    P(h+1:(floor(h/8)+1)*8,:) = 0;
end
if(mod(w, 8) ~=0)
    P(:,w+1:(floor(w/8)+1)*8) = 0;
end

% block
[h, w] = size(P);
block_h = h/8;
block_w = w/8;
k = 1;
result = [];
for i = 0:block_h-1
    for j = 0:block_w-1
        block = P(8*i + 1: 8*(i+1),8*j+1:8*(j+1)) - 128; % get the block
        D_block = dct2(block);
        D_block = round(D_block./QTAB); % quantation
        zigzag_D_idx = zig_zag(8);
        result(:,k) = D_block(zigzag_D_idx); % zigzag
        k = k + 1;
    end
end
\end{lstlisting}


\subsubsection{JPEG编码}
\par 上一步计算中得到了图像的DC系数和AC系数。定义函数\lstinline{dc_params.m}和\lstinline{ac_params.m}分别计算DC编码和AC编码。
\par DC编码可以分为两部分:huffman encode和binary encode。
\par DC编码的流程如下:
\begin{enumerate}
    \item 对DC系数进行差分。
    \item 对差分后的系数进行huffman encode和binary encode。
\end{enumerate}
\par 此处需要特别指出的是,0的binary encode不是'0',而是''。
\begin{lstlisting}[language=matlab, caption=DC code]
function [dc_code] = dc_encode(params)
    dc_params = params(1, :)';
    dc_params_diff = [];
    dc_params_diff(1) = dc_params(1);
    dc_len = length(dc_params);
    
    % diff, and code
    dc_code = strcat(huffman_encode(dc_params_diff(1)), category_encode(dc_params_diff(1)));
    for i = 2:dc_len
        dc_params_diff(i) = dc_params(i-1) - dc_params(i);
        huffman_code = huffman_encode(dc_params_diff(i));
        binary_code = binary_encode(dc_params_diff(i));
        dc_code = strcat(dc_code, huffman_code, binary_code);
    end
end
\end{lstlisting}

\par AC编码可以分为两部分:run/size encode和binary encode。
\par AC编码的流程较为复杂:
\begin{enumerate}
    \item 对系数进行zigzag扫描。
    \item 我们需要得出每一个非零系数的Size和Run,查表得到Huffman编码,再与Amptitude拼接。
    \item 如果遇到16个连零,我们需要将其编码为11111111001(ZRL);如果遇到结束,我们需要编码1010(EOB)。
\end{enumerate}

\begin{lstlisting}[language=matlab, caption=AC code]
function [ac_code] = ac_encode(params)
    ZRL = '11111111001';
    EOB = '1010';
    ac_code  = '';
    ac_params = params(2:end, :);
    for i = 1:size(ac_params, 2)
        vec = ac_params(:,i); % params after zigzag
        zero_num = 0;
        for j = 1:length(vec)
            if vec(j) ~= 0
                while zero_num >= 16
                    ac_code = strcat(ac_code, ZRL);
                    zero_num = zero_num - 16;
                end
                ac_code = strcat(ac_code, run_size_encode(zero_num, vec(j)));
                ac_code = strcat(ac_code, binary_encode(vec(j)));
                zero_num = 0;
            else
                zero_num = zero_num + 1;
            end
        end
        ac_code = strcat(ac_code, EOB); 
    end
end
\end{lstlisting}

\par 其中辅助函数如下:
\begin{lstlisting}[language=matlab, caption=huffman encode]
function [huffman_code] = huffman_encode(num)
    load('data/JpegCoeff.mat');
    if(num == 0)
        category = 0;
    else
        category = floor(log2(abs(num))) + 1;
    end
    
    huffman_code_length = DCTAB(category+1,1);
    huffman_code = DCTAB(category+1, 2:1 + huffman_code_length);
    huffman_code = num2str(huffman_code);
    huffman_code = strrep(huffman_code, ' ', '');
    
end
\end{lstlisting}

\begin{lstlisting}[language=matlab, caption=binary encode]
function [category_code] = binary_encode(num)
    load('data/JpegCoeff.mat');
    if(num == 0)
        category_code = '';
    else
        category_code = dec2bin(abs(num));
    end
    if(num < 0)
        for i = 1:length(category_code)
            category_code(i) = num2str(~str2num(category_code(i)));
        end
    end
end
\end{lstlisting}

\begin{lstlisting}[language=matlab, caption=run/size encode]
function [run_size_code] = run_size_encode(run, size)
    load('data/JpegCoeff.mat');
    if(size == 0)
        category = 0;
    else
        category = floor(log2(abs(size))) + 1;
    end
    idx = run*10 + category;
    code_length = ACTAB(idx,3);
    run_size_code = ACTAB(idx,4:4+code_length-1);
    run_size_code = num2str(run_size_code);
    run_size_code = strrep(run_size_code, ' ', '');
end
\end{lstlisting}

\par DC编码测例如下,与说明文档中相一致,可见结果正确:
\begin{figure}[ht]
     \centering
     \includegraphics[width=.4\textwidth]{result7.jpg}
     \caption{DC编码测试结果}
    \label{fig:result 7}
\end{figure}

\par AC编码测例如下,与说明文档中相一致,可见结果正确:
\begin{figure}[ht]
     \centering
     \includegraphics[width=.4\textwidth]{result8.jpg}
     \caption{AC编码测试结果}
    \label{fig:result 8}
\end{figure}

\par 为了方便起见,我们将分块、DCT、量化的算法封装为一个函数:
\begin{lstlisting}[language=matlab, caption=分块、DCT、量化]
function [result] = get_dct_matrix(P)
% fill the picture to block it
[h, w] = size(P);
if(mod(h, 8) ~=0)
    P(h+1:(floor(h/8)+1)*8,:) = 0;
end
if(mod(w, 8) ~=0)
    P(:,w+1:(floor(w/8)+1)*8) = 0;
end

% block
[h, w] = size(P);
block_h = h/8;
block_w = w/8;
k = 1;
result = [];
for i = 0:block_h-1
    for j = 0:block_w-1
        block = P(8*i + 1: 8*(i+1),8*j+1:8*(j+1)) - 128; % get the block
        D_block = dct2(block);
        D_block = round(D_block./QTAB); % quantation
        zigzag_D_idx = zig_zag(8);
        result(:,k) = D_block(zigzag_D_idx); % zigzag
        k = k + 1;
    end
end
end
\end{lstlisting}

\par 最后,我们将图片的DC码流、AC码流、长度、宽度写入文件:
\begin{lstlisting}[language=matlab, caption=写入文件]
load('data/hall.mat');
load('data/JpegCoeff');
P = double(hall_gray);

dct_matrix = get_dct_matrix(P);
dc_code = dc_encode(dct_matrix);
disp(length(dc_code));
ac_code = ac_encode(dct_matrix);
disp(length(ac_code));
[h, w] = size(P);
save('jpegcodes.mat', 'dc_code', 'ac_code', 'h', 'w');
\end{lstlisting}

\par 方便起见,我们定义函数实现jpeg编码,输入为图像数据和码流文件名称:
\begin{lstlisting}[language=matlab, caption=jpeg-encode.m]
function [] = jpeg_encode(original_picture, stream_name)
    load('data/JpegCoeff');
    dct_matrix = get_dct_matrix(original_picture);
    dc_code = dc_encode(dct_matrix);
    ac_code = ac_encode(dct_matrix);
    [h, w] = size(original_picture);
    save(stream_name, 'dc_code', 'ac_code', 'h', 'w');
end
\end{lstlisting}


\subsubsection{计算压缩比}
\par 由于原图的尺寸为20160 bytes,变换后DC码流长为2031,AC码流长为23072,总长度为25103 bits(3137 bytes),计算得出压缩比为6.4265。

\subsubsection{JPEG解码}

\par JPEG解码的基本步骤如下:
\begin{enumerate}
    \item DC解码和AC解码
    \par 考虑到DC码和AC码中的huffman code和run/size code均为huffman编码,我们可以构造huffman树。由于huffman编码较小,所以我们也可以考虑直接构造哈希表,在解码时直接检索即可:
    \begin{lstlisting}[language=matlab, caption=dc huffman tree]
function [huffman_tree_dict] = dc_huffman_tree_dict()
    load('data/JpegCoeff.mat');
    [code_num, len] = size(DCTAB);
    huffman_tree_dict = containers.Map();
    
    for i = 1:code_num
        % get huffman code
        huffman_code_length = DCTAB(i,1);
        huffman_code = DCTAB(i, 2:1 + huffman_code_length);
        huffman_code = num2str(huffman_code);
        huffman_code = strrep(huffman_code, ' ', '');
        disp([num2str(i-1), '->', huffman_code]);
        huffman_tree_dict(huffman_code) = i - 1;
    end
end
    \end{lstlisting}
    
    \begin{lstlisting}[language=matlab, caption=ac huffman tree]
function [huffman_tree_dict] = ac_huffman_tree_dict()
    load('data/JpegCoeff.mat');
    [code_num, len] = size(ACTAB);
    huffman_tree_dict = containers.Map();
    
    for i = 1:code_num
        % get huffman code
        huffman_code_length = ACTAB(i,3);
        huffman_code = ACTAB(i, 4:3 + huffman_code_length);
        huffman_code = num2str(huffman_code);
        huffman_code = strrep(huffman_code, ' ', '');
        disp([num2str(i-1), '->', huffman_code]);
        huffman_code = strcat('x', huffman_code);
        huffman_tree_dict(huffman_code) = i;
    end
end
    \end{lstlisting}
    
    \par 对于DC码流解码,哈希表构造完毕之后,我们首先将码流一一与哈希表对比,huffman code解码成功后即可得到binary code的长度,再对binary code进行解码即可得到原来的数值,完毕之后跳过binary code,进行解码下一个huffman code,直到码流的末尾。最后进行反差分即可。代码如下:
    \begin{lstlisting}[language=matlab, caption=dc decode]
function [dc_decode_data] = dc_decode(dc_stream)
    len = length(dc_stream);
    huffman_code_temp = '';
    huffman_code_dict = dc_huffman_tree_dict();
    dc_decode_data = [];
    i = 1;
    
    % diff decode
    while i <= len
        huffman_code_temp = strcat(huffman_code_temp, dc_stream(i));
        if(isKey(huffman_code_dict, huffman_code_temp))
            % get the decode data's length
            dc_decode_data_length = huffman_code_dict(huffman_code_temp);
            % get the binary code
            i = i + 1;
            if(dc_decode_data_length)
                binary_code = dc_stream(i:i + dc_decode_data_length - 1);
            else
                binary_code = '';
            end
            % decode data
            dc_decode_data = [dc_decode_data, binary_decode(binary_code)];
            % reset the huffman code
            huffman_code_temp = ''; 
            % jump
            i = i + dc_decode_data_length;
        else
            i = i + 1;
        end
    end
    
    % sum
    for i = 1:length(dc_decode_data) - 1
        dc_decode_data(i+1) = dc_decode_data(i) - dc_decode_data(i+1);
    end
end
    \end{lstlisting}
    
    \par 我们可以将编码之前的原始数据和解码之后的数据进行比较,发现我们可以完全复原其信息:
    \begin{figure}[ht]
        \centering
        \includegraphics[width=.4\textwidth]{result9.jpg}
        \caption{DC解码测试结果}
        \label{fig:result 9}
    \end{figure}
    
    \par 对于AC解码,我们首先针对每一个小块构建一个长为63的向量(注意除去了DC系数),将码流一一与哈希表对比,huffman code解码成功后即可得到run和size,再分别进行解码得到之前0的数目和数值,完毕之后跳过binary code,进行解码下一个huffman code,直到码流的末尾。同时要特别注意遇到EOB和ZRL的情况。
    
    \begin{lstlisting}[language=matlab, caption=ac decode]
function [ac_decode_data] = ac_decode(ac_stream)
    len = length(ac_stream);
    huffman_code_temp = '';
    huffman_code_dict = ac_huffman_tree_dict();
    ac_decode_data_vec = zeros(1,63);
    ac_decode_data = [];
    i = 1;
    idx = 1;
    
    ZRL = '11111111001';
    EOB = '1010';
    
    % diff decode
    while i <= len
        huffman_code_temp = strcat(huffman_code_temp, ac_stream(i));
        if(isKey(huffman_code_dict, huffman_code_temp))
            % get the decode data's length
            raw_data = huffman_code_dict(huffman_code_temp);
            binary_data_run = floor(raw_data / 10);
            binary_data_size = mod(raw_data, 10);
            while(binary_data_run > 0)
                idx = idx + 1;
                binary_data_run = binary_data_run - 1;
            end
            % get the binary code
            i = i + 1;
            if(binary_data_size)
                binary_code = ac_stream(i:i + binary_data_size - 1);
            else
                binary_code = '';
            end
            % decode data
            ac_decode_data_vec(idx) = binary_decode(binary_code);
            idx = idx + 1;
            % reset the huffman code
            huffman_code_temp = ''; 
            % jump
            i = i + binary_data_size;
        elseif(strcmp(huffman_code_temp, ZRL) == 1)
            for j = 1:16
                 idx = idx + 1;
            end
            i = i + 1;
            huffman_code_temp = '';
        elseif(strcmp(huffman_code_temp, EOB) == 1)
            ac_decode_data = [ac_decode_data; ac_decode_data_vec];
            ac_decode_data_vec = zeros(1,63);
            i = i + 1;
            idx = 1;
            huffman_code_temp = '';
        else
            i = i + 1;
        end
    end
    ac_decode_data = ac_decode_data';
end
    \end{lstlisting}
    
    \par 我们可以将编码之前的原始数据和解码之后的数据进行比较,发现我们可以完全复原其信息:
    \begin{figure}[ht]
        \centering
        \includegraphics[width=.4\textwidth]{result10.jpg}
        \caption{AC解码测试结果}
        \label{fig:result 10}
    \end{figure}
    
    \item 反zigzag扫描
    \par 反zigzag扫描的思路和zigzag扫描类似,只不过是将方向反过来:
    \begin{lstlisting}[language=matlab, caption=inv zig zag]
    function [y] = inv_zig_zag(vec)
        mat_size = sqrt(length(vec));
        zig_zag_idx = zig_zag(mat_size);
        y = zeros(1, mat_size);
        for i = 1:mat_size * mat_size
            y(zig_zag_idx(i)) = zig_zag_idx(i);
        end
        y = reshape(y, mat_size, mat_size);
        y = y';
    end
    \end{lstlisting}
    
    \item 反量化、逆DCT变换和图片拼接
    \begin{lstlisting}[language=matlab, caption=反量化、逆DCT变换和图片拼接]
        for i = 0:block_h - 1
        for j = 0:block_w - 1
            block = full_matrix(:,k);
            block = inv_zig_zag(block);
            block = block.*QTAB;
            block = idct2(block);
            block = block + 128;
            picture(8*i + 1: 8*(i+1),8*j + 1:8*(j+1)) = uint8(block);
            k = k + 1;
        end
    end
    \end{lstlisting}
    
\end{enumerate}

\par 方便起见,我们将解码封装为函数:
\begin{lstlisting}[language=matlab, caption=jpeg-decode]
function [picture] = jpeg_decode(stream_file)
    load(stream_file);
    load('data/JpegCoeff.mat');
    
    picture = zeros(h, w);
    % decode
    dc_matrix = dc_decode(dc_code);
    ac_matrix = ac_decode(ac_code);
    full_matrix = [dc_matrix; ac_matrix];
    disp(full_matrix);

    % inv zig zag & inv quat
    block_h = h/8;
    block_w = w/8;
    k = 1;

    for i = 0:block_h - 1
        for j = 0:block_w - 1
            block = full_matrix(:,k);
            block = inv_zig_zag(block);
            block = block.*QTAB;
            block = idct2(block);
            block = block + 128;
            picture(8*i + 1: 8*(i+1),8*j + 1:8*(j+1)) = uint8(block);
            k = k + 1;
        end
    end
end
\end{lstlisting}

\par 测试代码如下:
\begin{lstlisting}[language=matlab, caption=jpeg测试]
load('data/hall.mat')
subplot(1,2,1);
imshow(uint8(hall_gray));
title("original");

jpeg = jpeg_decode('jpegcodes.mat');
subplot(1,2,2);
imshow(uint8(jpeg));
title("jpeg");

mse = sum((double(jpeg) - double(hall_gray)).^2, 'all')/(h*w);
psnr = 10*log10(255*255/mse);
disp(psnr);
\end{lstlisting}

\par jpeg解码结果如下:
    \begin{figure}[ht]
        \centering
        \includegraphics[width=.8\textwidth]{result11.jpg}
        \caption{jpeg解码测试结果}
        \label{fig:result 11}
    \end{figure}

\par 可以看出,jpeg的恢复效果很好。计算得到图像的PSNR值为31.1874dB。

\subsubsection{将量化步长减小为原来的一半之后重新编解码}
\par 减小之后,PSNR值变为34.2067dB。DC码流长为2410,AC码流长为34164,总长度为36574 bits(4571 bytes)由于原图尺寸为20160 bytes,压缩比为4.4104。

\par 由此可见,图像PSNR值升高,压缩比降低,但仍然没有出现较大失真。

\begin{figure}[ht]
    \centering
    \includegraphics[width=.8\textwidth]{result12.jpg}
    \caption{量化步长减小后的jpeg解码测试结果}
    \label{fig:result 12}
\end{figure}

\subsubsection{雪花图像的编解码}

\par PSNR值为22.9244 dB。DC码流长为1403,AC码流长为43546,总长度为44949 bits(5618 bytes)由于原图尺寸为20160 bytes,压缩比为3.5885。图片的失真度较高。
\par 原因如下:jpeg编码适用于色彩变换较为平缓、高频分量较小的图片,而雪花图片的色彩分布没有规律可言,高频分量较大,因此在压缩之后失真较大。

\begin{figure}[ht]
    \centering
    \includegraphics[width=.8\textwidth]{result13.jpg}
    \caption{雪花图像的编解码}
    \label{fig:result 13}
\end{figure}

\subsection{信息隐藏}

\subsubsection{空域隐藏和提取}

\par 为了使得结果具有普适性,我们使用随机信息序列进行隐藏。我们将信息保存为二进制码流,然后储存至每个像素的最后一位。

\begin{figure}[ht]
    \centering
    \includegraphics[width=.8\textwidth]{result14.jpg}
    \caption{空域隐藏和提取}
    \label{fig:result 14}
\end{figure}

\begin{figure}[ht]
    \centering
    \includegraphics[width=.4\textwidth]{result15.jpg}
    \caption{空域隐藏和提取正确率}
    \label{fig:result 15}
\end{figure}

\par 可以看出,图片几乎无失真,但提取信息的正确率在50\%左右,这意味着其抗jpeg编码能力很差。

\subsubsection{变换域隐藏和提取}

\par 公平起见,我们仍然使用随机0/1序列进行测试。很显然可以猜测得到,方法2和3相比于方法1储存的信息少、图片失真率小。为了让方法2和方法3具有可比性,在方法2中我们在每一个块中只选取一个位进行储存信息,这样与方法3存储信息的密度相同。在方法2中为了比较不同替换策略的影响,此处我设置了三种替换策略:选取最大的量化值(2-1)、随机选取(2-2)、选取最小的量化值(2-3)。

\begin{description}
    \item[方法1:用信息位逐一替换掉每一个量化之后的DCT系数的最低位,再进行熵编码]
    
    \item[方法2:用信息位逐一替换掉若干量化之后的DCT系数的最低位,再进行熵编码]

    \item[方法3:将信息位追加在每一个块zig-zag顺序的最后一个非零DCT系数之后] 
    
\end{description}

\par 编码、解码的代码如下所示:

\begin{lstlisting}[language=matlab, caption=信息编码]
function [dc_code, ac_code, h, w, hidden_info] = jpeg_encode_with_info(original_picture, stream_name, QTAB, DCTAB, ACTAB, method)
    dct_matrix = get_dct_matrix_with_params(original_picture, QTAB);
    [dct_h, dct_w] = size(dct_matrix);
    hidden_info = uint8(randi([0 1], dct_h * dct_w ,1));
    hidden_info = reshape(hidden_info, dct_h, dct_w);
    [h, w] = size(original_picture);
    zig_zag_idx = zig_zag(8);
    m = median(QTAB, 'all');
    switch (method)
        case('method_1')
            dct_with_info = double(bitset(int64(dct_matrix), 1, hidden_info));
        case('method_2_1')
            [~, max_idx] = max(QTAB(zig_zag_idx));
            dct_with_info = dct_matrix;
            dct_with_info(max_idx,:) = double(bitset(int64(hidden_info(max_idx,:)), 1, hidden_info(max_idx,:)));
            hidden_info = hidden_info(max_idx,:);
        case('method_2_2')
            random_idx = 32;
            dct_with_info = dct_matrix;
            dct_with_info(random_idx,:) = double(bitset(int64(hidden_info(random_idx,:)), 1, hidden_info(random_idx,:)));
            hidden_info = hidden_info(random_idx,:);
        case('method_2_3')
            [~, min_idx] = min(QTAB(zig_zag_idx));
            dct_with_info = dct_matrix;
            dct_with_info(min_idx,:) = double(bitset(int64(hidden_info(min_idx,:)), 1, hidden_info(min_idx,:)));
            hidden_info = hidden_info(min_idx,:);
        case('method_3')
            dct_with_info = dct_matrix;
            hidden_info = int8(hidden_info(1:dct_w));
            hidden_info_tmp = hidden_info;
            for i = 1:length(hidden_info)
                if hidden_info_tmp(i) == 0  
                    hidden_info_tmp(i) = -1;
                end
            end
            idx = 1;
            for i = 1:dct_w
                for j = dct_h:-1:1
                    if(dct_with_info(j, i) ~= 0)
                        if (j ~= dct_h)
                            dct_with_info(j + 1, i) = hidden_info_tmp(idx);
                        else
                            dct_with_info(j, i) = hidden_info_tmp(idx);
                        end
                        break
                    end
                end
                idx = idx + 1;
            end
        otherwise
            ME = MException('available options: method_1/method_2_1/method_2_2/method_2_3/method_3');
            throw(ME);
    end
    dc_code = dc_encode(dct_with_info, DCTAB);
    ac_code = ac_encode(dct_with_info, ACTAB);
    save(stream_name, 'dc_code', 'ac_code', 'h', 'w');
end
\end{lstlisting}

\begin{lstlisting}[language=matlab, caption=信息解码]
function [picture, decode_info] = jpeg_decode_with_info(stream_file, QTAB, DCTAB, ACTAB, method)
    load(stream_file);
    
    picture = zeros(h, w);
    % decode
    dc_matrix = dc_decode(dc_code, DCTAB);
    ac_matrix = ac_decode(ac_code, ACTAB);
    full_matrix = [dc_matrix; ac_matrix];
    [dct_h, dct_w] = size(full_matrix);
    m = median(QTAB, 'all');
    zig_zag_idx = zig_zag(8);
    
    switch (method)
        case('method_1')
            decode_info = bitget(int64(full_matrix), 1);
        case('method_2_1')
            [~, max_idx] = max(QTAB(zig_zag_idx));
            decode_info = bitget(int64(full_matrix(max_idx,:)), 1);
        case('method_2_2')
            random_idx = 32;
            decode_info = bitget(int64(full_matrix(random_idx,:)), 1);
        case('method_2_3')
            [~, min_idx] = min(QTAB(zig_zag_idx));
            decode_info = bitget(int64(full_matrix(min_idx,:)), 1);
        case('method_3')
            decode_info = int8(zeros(dct_w, 1));
            for i = 1:dct_w
                for j = dct_h:-1:1
                    if(full_matrix(j,i) ~= 0)
                        if(full_matrix(j,i) > 0)
                            decode_info(i) = 1;
                        elseif(full_matrix(j,i) < 0)
                            decode_info(i) = 0;
                        end
                        break
                    end
                end
            end
        otherwise
            ME = MException('available options: method_1/method_2_1/method_2_2/method_2_3/method_3');
            throw(ME);
    end

    % inv zig zag & inv quat
    block_h = h/8;
    block_w = w/8;
    k = 1;

    for i = 0:block_h - 1
        for j = 0:block_w - 1
            block = full_matrix(:,k);
            block = inv_zig_zag(block);
            block = block.*QTAB;
            block = idct2(block);
            block = block + 128;
            picture(8*i + 1: 8*(i+1),8*j + 1:8*(j+1)) = uint8(block);
            k = k + 1;
        end
    end
end
\end{lstlisting}

\par 三种方法的压缩比、PSNR、准确率、主观评判质量如下表所示:

\begin{table}[htbp]
  \centering
  \caption{变换域隐藏结果}
    \begin{tabular}{cccccc}
          & 方法1   & 方法2-1 & 方法2-2 & 方法2-3 & 方法3 \\
    PSNR/dB & 15.5001 & 26.0098 & 29.5916 & 26.0061 & 28.9016 \\
    压缩比   & 2.8784 & 5.4431 & 6.0461 & 6.5612 & 6.1916 \\
    正确率   & 1     & 1     & 1     & 1     & 1 \\
    主观评判质量 & 差     & 中     & 中     & 好     & 好 \\
    \end{tabular}%
  \label{tab:addlabel}%
\end{table}%

\par 生成的图像如下所示:

\begin{figure}[ht]
    \centering
    \includegraphics[width=.9\textwidth]{result16.jpg}
    \caption{变换域隐藏图像}
    \label{fig:result 16}
\end{figure}

\par 由此可以得出,三种方法都能够正确解码隐藏信息。其中,方法2-3和方法3具有较好的效果:图像的PSNR值和压缩比均较高。

\subsection{人脸检测}

\subsubsection{人脸识别模型的训练}
\begin{description}
    \item [是否需要调整图片的大小]
    \par 不需要,因为我们只关注人脸的颜色。
    \item [L=3、4、5时,所得人脸特征v之间的关系]
    由于RGB值是一个uint8数值,所以L的值意味着提取RGB之中的前L位。L越大,标号相同的颜色就越少、所能表示的颜色就越精确、但数据量也越大。
    
\end{description}

\subsubsection{人脸检测算法}
\par 首先,我们需要将所有的R、G、B值拼接为一个值。同时为了权衡L的大小对识别的影响,我们也将提取位数L设置为参数:
\begin{lstlisting}[language=matlab, caption=RGB值的拼接]
function [val] = concact_rgb(r, g, b, l)    
    ignore_bits = 8 - l;
    r = bitshift(r, -ignore_bits, 'uint8');
    g = bitshift(g, -ignore_bits, 'uint8');
    b = bitshift(b, -ignore_bits, 'uint8');
    val = bitshift(r, l*2, 'uint32') + bitshift(g, l, 'uint32') + b;
end
\end{lstlisting}

\par 之后,我们使用拼接好的值计算概率密度:
\begin{lstlisting}[language=matlab, caption=概率密度计算]
function [destiny_list] = get_probability_density(img, l)
    [h, w, c] = size(img);
    img = reshape(img, h*w, c);
    destiny_list = zeros(1, 2^(3 * l));
    for i = 1:h * w
        feature = concact_rgb(img(i, 1), img(i, 2), img(i, 3), l); 
        destiny_list(feature + 1) = destiny_list(feature + 1) + 1;
    end
    destiny_list = destiny_list/(h*w);
end 
\end{lstlisting}

\par 一个概率密度提取示例如下,从上到下分别为l = 3、4、5的情形:

\begin{figure}[ht]
    \centering
    \includegraphics[width=.9\textwidth]{result17.jpg}
    \caption{概率密度提取示例}
    \label{fig:result 17}
\end{figure}

\par 可以较为容易地看出,l越大,数据量越大,但对于颜色的分辨能力就越精确。

\par 接下来就可以准备训练特征了。
\begin{description}
    \item [训练标准特征] 
    计算训练集中人脸特征的平均值。由于训练集中的图片几乎全部区域均为人脸,所以我们无需手动定位。计算公式如下:
    \begin{equation}
        \mathbf{v} = \frac{1}{I}\Sigma_i\mathbf{u}(R_i)
    \end{equation}
    \item [分块]
    将图像划分为小块分别与标准特征比较。
    \item [计算与人脸标准的距离]
    定义待定特征和人脸标准的距离如下:
    \begin{equation}
        d(\mathbf{u}(R),\mathbf{v}) = 1 - \rho(\mathbf{u}(R),\mathbf{v}) = 1 - \Sigma_n\sqrt{u_n v_n}
    \end{equation}
    \item [比较]
    判断待定特征和人脸标准的相似程度,如果相似,则使用锚定框标记:
    \begin{equation}
        R\in Face \quad \quad if \quad d(\mathbf{u}(R),\mathbf{v}) < \epsilon
    \end{equation}
\end{description}

\par 训练代码如下:
\begin{lstlisting}[language=matlab, caption=特征提取]
function [destiny_list] = extract_face_feature(l)
    destiny_list = zeros(1, 2^(3 * l));
    for i = 1:33
        picture = imread(strcat('data/Faces/', num2str(i), '.bmp'));
        destiny_list_single = get_probability_density(picture, l);
        destiny_list = destiny_list + destiny_list_single;
    end
    destiny_list = destiny_list/33;
end
\end{lstlisting}

\par 检测代码如下:
\begin{lstlisting}[language=matlab, caption=人脸检测]
function [] = get_face_area(file_name, ref_destiny, l, threshold, block_size)
    % get the feature
    subplot(2,1,1);
    idx = 1:length(ref_destiny);
    plot(idx, ref_destiny);

    % split the picture
    P = imread(file_name);

    subplot(2,1,2);
    imshow(P);

    % block
    [h, w, c] = size(P);
    block_h = floor(h/block_size);
    block_w = floor(w/block_size);
    title(strcat('threshold=',num2str(threshold), ' l=', num2str(l), ' block-size=', num2str(block_size)));

    for i = 0:block_h-1
        for j = 0:block_w-1
            block = P(block_size * i + 1: block_size * (i + 1),block_size * j + 1:block_size * (j + 1),:); % get the block
            hypo_destiny = get_probability_density(block, l);
            if(1 - sum(sqrt(hypo_destiny).*sqrt(ref_destiny)) < threshold)
                rectangle('position',[block_size*j + 1,block_size*i + 1,block_size,block_size],'EdgeColor','r');
            end
        end
    end
end
\end{lstlisting}

\par 选取若干测试图片(\ref{fig:result 18}、\ref{fig:result 19}、\ref{fig:result 20}),可以发现该检测方法的一些问题:
\begin{enumerate}
    \item 基本可以识别出人脸特征。
    \item 容易将手或与皮肤颜色相近的背景识别为人脸。
    \item 对于黑色人种的识别效果较差。
\end{enumerate}

\begin{figure}[ht]
    \centering
    \includegraphics[width=.7\textwidth]{result18.jpg}
    \caption{识别样例1}
    \label{fig:result 18}
\end{figure}

\begin{figure}[ht]
    \centering
    \includegraphics[width=.7\textwidth]{result19.jpg}
    \caption{识别样例2}
    \label{fig:result 19}
\end{figure}

\begin{figure}[ht]
    \centering
    \includegraphics[width=.7\textwidth]{result20.jpg}
    \caption{识别样例3}
    \label{fig:result 20}
\end{figure}

\par 之后我们选取一张合照,并寻找最适合其的超参。对一张合照枚举l(3、4、5)、threshold(0.6、0.7、0.8、0.9)、block size(10、20)三个值,得到的结果如图(\ref{fig:result 21}、\ref{fig:result 22}):

\begin{figure}[ht]
    \centering
    \includegraphics[width=.9\textwidth]{result21.jpg}
    \caption{合照人脸检测1}
    \label{fig:result 21}
\end{figure}

\begin{figure}[ht]
    \centering
    \includegraphics[width=.9\textwidth]{result22.jpg}
    \caption{合照人脸检测2}
    \label{fig:result 22}
\end{figure}

\par 可以得出以下结论:
\begin{enumerate}
    \item l越大,方块被识别为人脸的概率就越大。
    \item threshold越大,方块被识别为人脸的概率就越大。
    \item block size应该与人脸的尺寸相近或略小。
\end{enumerate}

\par 经过比对,我选出了效果较好的参数组,其中第一组的效果如图\ref{fig:result 23}:
\begin{enumerate}
    \item threshold = 0.6, l = 4, block-size = 10
    \item threshold = 0.7, l = 5, block-size = 20
    \item threshold = 0.8, l = 5, block-size = 10
\end{enumerate}

\begin{figure}[ht]
    \centering
    \includegraphics[width=.9\textwidth]{result23.jpg}
    \caption{threshold = 0.6, l = 4, block-size = 10}
    \label{fig:result 23}
\end{figure}

\subsubsection{图像变换后的人脸检测}
\par 此处仍选择之前的最佳超参:threshold = 0.6, l = 4, block-size = 10。测试代码如下:
\begin{lstlisting}[language=matlab, caption=图像变换]
l = 4;
block_size = 10;
threshold = 0.6;

P = imread('data/test_3.bmp');

P_2 = rot90(P);
get_face_area(P_2, ref_destiny, l, threshold, block_size);
title('rotate 90 degrees');

[h, w, c] = size(P);
P_3 = imresize(P, [h, w*2],'nearest');
get_face_area(P_3, ref_destiny, l, threshold, block_size);
title('enlarge width');

P_4 = imadjust(P,[.2 .3 0; .6 .7 1],[]);
get_face_area(P_4, ref_destiny, l, threshold, block_size);
title('adjust the color');
\end{lstlisting}

\begin{description}
    \item [图片顺时针旋转90度] 见图\ref{fig:result 24}
\begin{figure}[ht]
    \centering
    \includegraphics[width=.9\textwidth]{result24.jpg}
    \caption{图片顺时针旋转90度}
    \label{fig:result 24}
\end{figure}
    \item [宽度拉伸为之前的2倍] 见图\ref{fig:result 25}
\begin{figure}[ht]
    \centering
    \includegraphics[width=.9\textwidth]{result25.jpg}
    \caption{宽度拉伸为之前的2倍}
    \label{fig:result 25}
\end{figure}
    \item [改变图片颜色] 见图\ref{fig:result 26}
\begin{figure}[ht]
    \centering
    \includegraphics[width=.9\textwidth]{result26.jpg}
    \caption{改变图片颜色}
    \label{fig:result 26}
\end{figure}
\end{description}

\par 可以得出以下结论:旋转和改变尺寸,对识别效果影响不算大,但改变图片颜色对于识别效果影响巨大。这是因为我们的判别准则是基于颜色进行的。

\subsubsection{重新选择人脸样本训练标准}

\par 综合上述实验,我认为更合适的人脸样本训练标准如下:
\begin{enumerate}
    \item 样本应该涵盖更多的人种。
    \item 由于以上识别方法容易把身体的其它部位或背景颜色识别为人脸,我们应该制定一个与轮廓有关的识别算法,使用轮廓和颜色进行综合判断。
    \item 样本应该涵盖多种光照角度下的情况,使得特征更具有鲁棒性。
\end{enumerate}

\end{document}